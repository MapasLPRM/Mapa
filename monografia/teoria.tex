\section{Fundamentos de Mapas Geográficos}
	\subsection{Coordenadas E Zoom} usar texto da figura do livro do google
	\subsection{Marcadores e Icones} pag 140
	\subsection{InfoWindow} videos e fotos pagina 152
	\subsection{Polylines}

\section{Estratégias para lidar com muitos marcadores}
  \subsection{Reduzir numero de marcadores}
	\subsubsection{Pesquisa}
	\subsubsection{Filtro}
	\subsubsection{Otimizar a representação}
nem sempre usar marcadores para representar cada parte de um elemento... linhas nao precisam de marcadores para cada vertice , e grupos proximos podem de ser representados por unico poligono.
 pagina 198
 
 (pagina 88 ,silva, tabela 6, figura 93, figura 65 ())

  \subsection{Agrupamento/Clustering}
  pagina 199 (google)
		\subsubsection{Por grelha}
		\subsubsection{Por distancia}
		\subsubsection{Por região}
		
\section{Algorítimos de agrupamento}
	\subsection{Métodos baseados em grelha}
		\subsubsection{WaveCluster}
	\subsection{Aplicações}
		\subsubsection{MarkerCluster}
		pagina 206 a 212
		\subsubsection{MarkerClustered}



\section{Planilhas eletrônicas e Mapas}
\subsection{O desafio chinês}
mostra o uso de planilhas pelo governo chines \cite{chinaPlanilha}
\subsection{Domínios de conhecimento}
Mostra a importância do uso de planilhas \cite{credinePlanilha} 
\subsection{Usando planilhas como fonte de dados para Mapas Geográficos}
Mostra \cite{lieberman2009spatio}