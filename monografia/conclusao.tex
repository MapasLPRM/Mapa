Este capítulo apresenta as conclusões do trabalho realizado, mostrando suas contribuições. Por fim, são apresentadas suas limitações e perspectivas de trabalhos futuros.

\section{Conclusões}
Com a popularização dos computadores e da internet, está acontecendo um aumento da interação social entre as pessoas no meio virtual. Essa interação permite o cultivo de novos comportamentos e interações sociais como o crowdsourcing. Além disso, tem crescido a necessidade de mecanismos virtuais que permitam as pessoas exercer sua cidadania de forma conectada e contribuir para a cidade com  criticas e sugestões. Entretanto, tais mecanismos precisam lidar com uma enorme quantidade de informação e isso dificulta a visualização dos dados coletados.

Neste contexto, este trabalho apresentou a criação do framework Searchlight com o intuito de facilitar a visualização de informações crowdsourcing em Mapas Web. Pesquisou-se por estratégias para resolver os problemas de sobreposição de informações, zoom arbitrário e informações arbitrárias. E para solucionar esses problemas, foi necessário utilizar as estratégias de redução de número de marcadores e agrupamento de marcadores.

Para a aplicar essas estratégias, o framework Searchlight implementou alguns recursos como o filtro por categorias, o agrupamento de marcadores, Baloes de Resumo e foco em grupo. Esses recursos, ajudaram a melhorar a visualização do mapa e a deixá-lo mais limpo e compreensível.

Durante a fase de pesquisa, percebeu-se, que além de melhorar a visualização do mapa, era necessário  melhorar também o acesso ao mapa. Para isso, foi feito a implementação do recurso de gerar e compartilhar mapas automaticamente. Com esse recurso, usuários que não possuem conhecimento de programação também podem criar  e compartilhar os seus próprios mapas.

\section{Dificuldades Encontradas}
No desenvolvimento do framework Searchlight foram encontradas dificuldades que podem ser resumidas em 3 áreas distintas: Design, Compatibilidade e Detecção de Erros.

Na área de Design a maior dificuldade foi decidir qual seria os objetos de interface para interagir com o usuário. Por exemplo, para exibir os balões de resumo inicialmente pensou-se em utilizar o botão direito do mouse e o botão esquerdo para expandir o zoom do grupo. Porém, isso ia de encontro com a utilização do framework por tablets e smartphones, pois estes aparelhos não possuem a opção de click com botão direito do mouse. 

Para resolver esse problema, adotou-se um click para abrir o balão de resumo e duplo click para realizar zoom em grupo. Isso aumentou a complexidade do controle de interface, pois foi necessário criar algumas variáveis de controle para tratar o toque e duplo toque no contexto de tablets e smartphones.

Na área de Compatibilidade, a maior dificuldade foi com as diferentes regras e políticas de segurança adotadas pelos navegadores. Por exemplo, o firefox não permite acesso direto aos arquivos JSON quando o pagina html é acessada localmente via protocolo file. Porém, se a pagina é acessada localmente mas via protocolo http o firefox libera o acesso. 

Também ocorreram dificuldades para exibir os balões de resumo nos navegadores Firefox e Internet Explorer. Para solucionar esse problema, foi preciso definir um largura fixa para o balão de resumo que aparece nesses navegadores.
 
De todas as dificuldades encontradas a que mais atrapalhou o desenvolvimento do projeto foi a detecção de erros. JavaScript é uma linguagem de script  e por causa disso boa parte dos erros acontecem durante a execução do código. Isso dificulta o desenvolvimento pois é difícil localizar o que provocou o erro. Em muitos casos, as mensagens de erro são consequências de um erro que não é exibido pelo navegador. Por causa disso, pode-se perder dias procurando um erro que em uma linguagem compilada seria detectado no momento da compilação.


\section{Limitações e Perspectivas Futuras}


No cenário alcançado, algumas limitações são observadas, o que dá margem para a realização de trabalhos futuros. Dentre elas destaca-se o fato da geração automática de mapas aceitar apenas planilhas do Google Docs e não ser possível utilizar planilhas de aplicativos como Excel, Calc e Numbers. Atualmente, o usuário pode adicionar seus dados nesses outros formatos de planilhas, mas para usa-las no mapa  é necessário que elas sejam convertidas para  o formato de planilha do Google Docs.

Além disso, na \autoref{sec-estrategias} foi demostrado que uma das estratégias para se reduzir o número de marcadores é a Otimização visual. Um exemplo de otimização visual genérica que poderia ser implementada no framework é a conexão de marcadores de uma mesmas categoria. Por exemplo, se consideramos um mapa que exibe todos os pontos de ônibus de uma cidade e conectarmos os pontos pertencem a uma mesma linha de ônibus, podemos substituir todos os marcadores do trajeto por um desenho de uma linha passando pela localização dos pontos de ônibus. 

Ainda assim,há espaço para acréscimos de novas funcionalidades  como: filtros por data em mapas temporais, campos de busca em mapas com muita informação textual, e muitas outras.  

Portanto, vislumbra-se a possibilidade de trabalhos futuros objetivando o oferecimento de mais serviços por parte do framework Searchlight e uma melhoria continua dos recursos de visualização de mapas de crowdsourcing.
